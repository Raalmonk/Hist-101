\documentclass[11pt,a4paper]{article}
\usepackage[utf8]{inputenc}
\usepackage[T1]{fontenc}
\usepackage{geometry}
\geometry{margin=1in}
\usepackage{hyperref}
\usepackage{amsmath,amssymb}
\usepackage{tikz}

\title{Notes on Early Rome \\ \small “Rome’s First Centuries”}
\author{}
\date{}

\begin{document}
\maketitle
\tableofcontents
\bigskip

\section[Origins and Urbanization (8th--6th c. B.C.)]{Origins and Urbanization (8th–6th c.\,B.C.)}
\begin{itemize}
  \item \textbf{Geography:} Seven hills on Tiber’s bank; salt road + coastal highway converge.
  \item \textbf{Early Settlements:} Clusters of huts on hilltops; small cemeteries; Middle Bronze Age pottery.
  \item \textbf{Forum Formation (c.\,650–600\,B.C.):}
    \begin{itemize}
      \item Drain marshy valley, lay beaten-earth paving.
      \item Site for public assemblies and rituals.
    \end{itemize}
  \item \textbf{Regia \& Comitium:} 
    \begin{itemize}
      \item Regia (king’s religious seat) by 625\,B.C.
      \item Comitium (public meeting space) with proto–Curia Hostilia.
    \end{itemize}
  \item \textbf{Elite Houses:} \textit{Atrium}, \textit{atria}, enclosed gardens appear by 550\,B.C.
\end{itemize}

\section{Writing Early History}
\begin{itemize}
  \item \textbf{Sources:} 
    \begin{itemize}
      \item Pontifical annals (year–by–year priestly records, fames, earthquakes, prodigies).
      \item Greek precedents: Fabius Pictor (c.\,200\,B.C.), Cato the Elder (\textit{Origines}).
      \item Surviving authors: Cicero, Livy; later Greek (Diodorus, Plutarch, Cassius Dio).
    \end{itemize}
  \item \textbf{Myth \& Invention:}
    \begin{itemize}
      \item Embellishments plausible, character-driven, non-contradictory.
      \item Draw the past into the author’s own values.
    \end{itemize}
  \item \textbf{Archaeology \& Inscriptions:} Combine with literary tradition to recover 6th–5th c.\ outline.
\end{itemize}

\section{Roman Kingship}
\subsection{Political \& Religious Offices}
\begin{itemize}
  \item \textit{Rex} = military, judicial \& religious head.
  \item \textit{Rex sacrorum}: priestly successor to king’s cult functions after monarchy ends.
  \item \textbf{Senate:}  
    \begin{itemize}
      \item Council of elders; chose kings, advised \& checked them.
      \item Met in Curia Hostilia (\(\sim\)600\,B.C.).
    \end{itemize}
\end{itemize}

\subsection{Social Organization}
\begin{itemize}
  \item \textbf{Curiae (30)} assemble into three tribes: Tities, Ramnes, Luceres.
  \item \textit{Curiae} perform communal rites; officials \textit{curiones}, \textit{libones}, \textit{flamines}.
  \item \textbf{Servian Reforms:}
    \begin{itemize}
      \item Census by wealth \& residence (\textit{classis} vs.\ \textit{infra classem}).
      \item \textit{Centuriae}: subdivisions by class; later military units and voting blocs.
    \end{itemize}
  \item Territorial \textit{tribes}: 4 urban + expanding rural subdivisions for mustering.
\end{itemize}

\section{Transition to Republic (c.\,509\,B.C.)}
\begin{itemize}
  \item \textbf{Expulsion of Tarquins:}
    \begin{itemize}
      \item Sextus Tarquinius \(\to\) rape of Lucretia \(\to\) her suicide.
      \item Brutus, Collatinus, Publicola seize power; end of kings.
    \end{itemize}
  \item \textbf{Magistracies:}
    \begin{itemize}
      \item Annual, collegial offices (Consuls, Praetors, etc.).
      \item Limit: one–year term; multiple holders to check abuses.
    \end{itemize}
  \item \textbf{Dictator:}
    \begin{itemize}
      \item Appointed by consul in emergencies (6 months max).
      \item Magister equitum as deputy.
    \end{itemize}
  \item \textbf{Assemblies:}
    \begin{itemize}
      \item \textit{Comitia Curiata} (30 curiae): ritual/witness functions.
      \item \textit{Comitia Centuriata}: by centuriae, elect consuls, judge capital cases.
      \item \textit{Comitia Tributa}: by tribes, elect lower magistrates, local legislation.
    \end{itemize}
  \item \textbf{Twelve Tables (450\,B.C.):}
    \begin{itemize}
      \item Decemviri draft core private law: family, inheritance, property, debt, procedure.
      \item Public display \(\to\) legal transparency.
    \end{itemize}
\end{itemize}

\section{Rome and the Latins}
\begin{itemize}
  \item \textbf{Myth of Aeneas \& Alba Longa}: shared origin story.
  \item \textit{Latiar}: annual festival on Alban Mount for Jupiter Latiaris.
  \item Common cult sites: Lavinium (Penates), Aricia (Diana’s grove), Aventine (Diana).
  \item \textbf{Shared Rights:}
    \begin{itemize}
      \item \textit{Conubium} (intermarriage), \textit{Commercium} (trade contracts), \textit{Ius migrationis}.
    \end{itemize}
  \item \textbf{No political unity}: city–state rivalry; Rome’s ascendancy via conquest \& Carthaginian treaty (\(\sim\)500\,B.C.).
\end{itemize}

\section{Rome in the 5th Century B.C.}
\begin{itemize}
  \item \textbf{External threats:}
    \begin{itemize}
      \item Highland raiders (Sabines, Volsci, Aequi) and Greek city defeats (Messapii, Samnites).
      \item Latin cities fall \(\to\) re-colonized as \textit{coloniae}.
    \end{itemize}
  \item \textbf{Legendary exemplars:}
    \begin{itemize}
      \item Coriolanus: exile–turned–enemy, spared by Roman matron intercession.
      \item Cincinnatus: dictator for 16 days, victorious \& returned to plow.
    \end{itemize}
  \item \textbf{Struggle of the Orders:}
    \begin{itemize}
      \item Patrician–plebeian conflict over offices, punishments, legislation.
      \item Plebeian \textit{secessio}; creation of \textit{tribuni plebis} (sacrosanct, \textit{auxilium}).
    \end{itemize}
\end{itemize}

\end{document}
